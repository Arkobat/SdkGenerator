\section{Implementation \label{sec:implementation}}

\subsection{Schema Provider}
For this prototype, only a single Schema Provider has been implemented, which is a C\# Schema Provider.
This will follow the principles defined in Section \ref{sec:design:schema_provider}.
One of the main features is the mapping of built-in types.
As the definition does not specify a built-in type for all the types provided in C\#, multiple types have been mapped to the same built-in type.
An example of this is \textit{byte}, \textit{short} and \textit{int} all being mapped to a \textit{int32}.
A full example of how types are mapped can be viewed in Listing \ref{lst:dto_type_provider}, where the mapping to \textit{int32} is highlighted. 
\begin{lstlisting}[caption={C\# type schema mapping}, label={lst:dto_type_provider}, style=base_csharp]
dtoDirectory
    .AddMapping<bool>("boolean")
    .AddMapping<char>("char")
    .AddMapping<string>("string")
     // ------------
    .AddMapping<byte>("int32")
    .AddMapping<sbyte>("int32")
    .AddMapping<short>("int32")
    .AddMapping<ushort>("int32") 
    .AddMapping<decimal>("int32")
    .AddMapping<int>("int32")
    .AddMapping<uint>("int32")
    .AddMapping<nint>("int32")
    .AddMapping<nuint>("int32")
     // ------------
    .AddMapping<long>("int64")
    .AddMapping<ulong>("int64")
    .AddMapping<float>("float")
    .AddMapping<double>("double")
    .AddMapping<Guid>("guid")
    .AddMapping<DateOnly>("dateOnly")
    .AddMapping<DateTime>("date")
    .AddMapping<DateTimeOffset>("dateTimeOffset")
    .AddMapping(typeof(List<>), "List")
    .AddMapping(typeof(HashSet<>), "List")
    .AddMapping(typeof(Dictionary<object, object>), "Map");
\end{lstlisting}
\noindent
Other than the mapping, the implementation does not have anything special, that is not discussed in section \ref{sec:design:schema_provider}.
The program converts on class to a schema at a time and adds it to an internal storage.
Afterwards, all associated classes are converted the same way.
All the mapping is done according to the specification of the schema, with properties either being a built-in type, as mapped in Listing \ref{lst:dto_type_provider}, or another schema.
Once all classes have been mapped, the list of all schemes can be extracted from the storage, in serialized.

\subsection{Schema Consumer}
The schema consumer is designed to allow adding multiple languages with minimum code.
By limiting the unique amount of work required for each language to be implemented, the support is easily
This is done by handling as much as the transformation behind the scenes, and only having a few interfaces that would need to be implemented to add support for a new language.

\subsubsection{Built in types \label{sec:implementation:built_in_types}}
To accommodate built-in types having different formats in different programming languages, it is possible to register custom converters. Listing \ref{lst:dto_type_consumer} shows how all built-in types are registered.
\begin{lstlisting}[caption={Schema types to C\# mapping}, label={lst:dto_type_consumer}, style=base_csharp]
dtoDirectory
    .Register<object>("object")
    .Register<bool>("boolean")
    .Register<char>("char")
    .Register<string>("string")
    .Register<int>("int32")
    .Register<long>("int64")
    .Register<float>("float")
    .Register<double>("double")
    .Register<Guid>("guid")
    .Register<DateOnly>("dateOnly")
    .Register<DateTime>("date")
    .Register<DateTimeOffset>("dateTimeOffset")
    .Register(typeof(List<>), "List")
    .Register(typeof(Dictionary<object, object>), "Map");
\end{lstlisting}
All registered types require a custom Type Converter, which is shown in Listing \ref{lst:ITypeConvertor}.
\begin{lstlisting}[caption={ITypeConvertor interface}, label={lst:ITypeConvertor}, style=base_csharp]
public interface ITypeConvertor<T> : ITypeConvertor
{
    Type ITypeConvertor.TypeToConvert() => typeof(T);
    public string ConvertProperty(SchemaProperty context);
}
\end{lstlisting}
\noindent
The implementation of the type converter, tells the system how to format a property for the target language.
An example of the need for this interface is when e.g. converting an \textit{int32}. In Java, the property type would be \textit{int}, while it would be \textit{number} in TypeScript.

\subsubsection{Class Converter}
Another one of the main features of the solution is the implementations of the \textit{IClassConverter} interface, which can be seen in Listing \ref{lst:class_converter}.
This class serves as the main element to convert a schema to a complete SDK.

\begin{lstlisting}[caption={IClassConvertor interface}, label={lst:class_converter}, style=base_csharp]
public interface IConvertor
{
    public string TargetLanguage { get; }
}

public interface IClassConvertor : IConvertor
{
    /// <summary>
    /// Defines the template from which a class template is created from.
    /// Placeholders, conditions and more are supported via <see href="https://github.com/scriban/scriban">Scriban</see>.
    /// </summary>
    /// <returns>A complete class template.</returns>
    public string GetClassTemplate();

    /// <summary>
    /// Formats a class property, to be ready to add into the class.
    /// The output will be available when parsing the class template.
    /// </summary>
    /// <param name="classProperty">The details for the property.</param>
    /// <returns>A formatted property.</returns>
    public string FormatProperty(ClassProperty classProperty);

    /// <summary>
    /// An optional way to further transform a class template, after the system has performed the initial transformation.
    /// </summary>
    /// <param name="classTemplate">The pre parsed template</param>
    /// <returns>An updated template</returns>
    public ClassTemplate PostTransform(ClassTemplate classTemplate)
    {
        return classTemplate;
    }
}
\end{lstlisting}
\noindent
To better understand how this interface is used, an in-depth explanation, with examples, is provided.
The main functionality lies in the following three methods.
\begin{itemize}
    \item \textbf{GetClassTemplate}: This provides a class template, that when populated will create an entire class for the specific language.
    This is the one of the core functions that drives the entire solution.
    The code in Listing \ref{lst:java_class_converter_getclasstemplate} shows an example of a Java class.
    \begin{lstlisting}[caption={Implementation of a Java class template}, label={lst:java_class_converter_getclasstemplate}, style=base_csharp]
    package {{namespace}};
    
    import {{import}};
    
    public class {{class}} {
        
        {{property}};
    
        
        {{method}};
    
    }
    \end{lstlisting}

    \item \textbf{FormatProperty}: Instead of having the class template containing checks properties, a separate method is instead used to parse these properties.
    This should return a correctly formatted property, which can be directly inserted into the class template.
    Listing \ref{lst:java_class_converter_formatproperty} shows an example on how Java properties are formatted.
    \begin{lstlisting}[caption={How a Java property is defined}, label={lst:java_class_converter_formatproperty}, style=base]
$"{(classProperty.Nullable ? "@Nullable" : "@NotNull")}\n\t" +
$"private {classProperty.Type} {classProperty.Name};";
    \end{lstlisting}
    \noindent

    \item \textbf{PostTransform}: The last method is an optional way to transform the Class Template further than is done behind the scenes, and create custom data for the specific language.
    This is e.g.\ used in Java to create getters, which is not needed in either TypeScript or C\#.
\end{itemize}

\subsection{Diff Checker}
When implementing the Difference Checker, a Diff Schema is needed. 
The purpose of this schema is to return all changes to the user. 
The checker will iterate over all schemes, returning a list of all changes.
It will differentiate objects by the key of the schema. This means, that if a key name has been modified, the system will treat it as if the old schema has been removed, and a new one has been added.
A full example of how the diff schema is defined can be viewed in Listing \ref{code:diff_example}.
\begin{lstlisting}[caption={Diff schema example}, label={code:diff_example}, style=yaml]
- type: "Removed"
  namespace: "Example.Model"
  name: "TestClass01"
- type: "Added"
  namespace: "Example.Model"
  name: "TestClass02"
- type: "Modified"
  namespace: "Example.Model"
  name: "TestClass03"
  differences:
    - path: "abstract"
      oldValue: "false"
      newValue: "true"
    - path: "generics"
      oldValue: "[TKey, TValue]"
      newValue: "[TKey, TVValue, TDefault]"
  properties:
    - type: "Removed"
      name: "MyProp1"
    - type: "Added"
      name: "MyProp2"
    - type: "Modified"
      name: "MyProp3"
      differences:
        - path: "type"
          oldValue: "string"
          newValue: "int32"
        - path: "nullable"
          oldValue: "true"
          newValue: "false"
\end{lstlisting}